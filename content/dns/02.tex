\newpage

\section{Aufgabe 1: f}

\subsection{Aufgabenstellung}
Machen Sie sich zunächst mit dem Befehl und den Parametern vertraut. (0,5 Punkte)

\subsection{Vorbereitung}
Für diese Aufgabe sollte man sich eine gute Wiki-Seite suchen, in unserem Fall benutzen wir die von ubuntuusers (\url{https://wiki.ubuntuusers.de/netstat/}).

\subsection{Durchführung}
netstat ist ein Diagnose-Werkzeug um Informationen über die Netzwerkschnittstelle zu bekommen.
\textbf{Syntax}: 
\begin{lstlisting}
$ netstat [OPTIONEN]
\end{lstlisting}
\textbf{Parameter}: 
\begin{table}[H]
	\tablestyle
	\begin{tabular}{lll}
		\toprule
			Option & Abkürzung & Beschreibung \tabularnewline
				
		\midrule
			- -tcp & -t & Zeigt nur TCP-Sockets an\tabularnewline
			- -udp & -t & Zeigt nur UDP-Sockets an\tabularnewline
			- -listening & -t & Zeigt nur offene Ports an\tabularnewline
			- -numeric & -n & Übersetzt keine Nummern in Namen\tabularnewline
			- -extend & -e & Zeigt erweiterte Informationen an\tabularnewline
			- -program & -p & Zeigt die Prozess-ID und den Programmnamen des Prozesses an\tabularnewline
			
	\end{tabular}
\end{table}
 
\subsection{Fazit}
Der \textit{netstat} ist ein simples Kommando und sollte niemanden schwerfallen.

\newpage

\section{Aufgabe 1: g}

\subsection{Aufgabenstellung}
Starten Sie nun einen Web-Browser. Versuchen Sie eine URL (z.B. www.google.de) zu
öffnen, so wird eine Netzwerk-Verbindung zwischen Ihrem und dem entfernten Rechner
aufgebaut, was sich in einer entsprechenden Ausgabe manifestiert. Machen Sie sich mit
dem Ausgabe-Format vertraut und ermitteln Sie die Adressinformationen (IP-Adresse /
Port) der Kommunikationspartner bei dem Zugriff auf Google! (1 Punkt)

\subsection{Vorbereitung}
Für diese Aufgabe muss man sicher vorher mit den Ausgabe-Format vertraut machen.

\subsection{Fazit}
Wir konnten diese Aufgabe nicht lösen, da wir nicht verstanden haben was mit dem Ausgabe-Format gemeint ist und wie man nur den Zugriff auf Google sehen kann.

\section{Aufgabe 1: h}

\subsection{Aufgabenstellung}
Der netstat Befehl zeigt auch den Zustand einer Netzwerkverbindung an. Bei einem
Verbindungsaufbau zwischen Client und Server durchläuft die Verbindung verschiedene
Zustände. Ist die Verbindung aufgebaut, so ist der Zustand VERBUNDEN (bzw. ESTABLISHED).
In der Regel werden die Zustände so schnell durchlaufen, dass die Zustände nicht sichtbar

werden. Die Anzeige der Zwischenzustände kann nur über einen Trick sichtbar gemacht:
versucht man einen Port auf dem Server zu kontaktieren, der von diesem nicht durch eine
entsprechenden Service unterstützt wird, so wird ein Zwischenzustand sichtbar. Überlegen
Sie sich einen Aufruf, mit dem Sie den Zwischenzustand beim Versuch eines Verbindungsaufbaus sichtbar machen können! (1 Punkt)

\subsection{Vorbereitung}
Für diese Aufgabe gibt es keine Vorbereitung.

\subsection{Fazit}
Wir konnten diese Aufgabe nicht lösen, da wir nicht wissen, wie wir hätten diesen Zwischenstand speichern oder erzeugen können.
