\newpage

\section{Aufgabe 1: f}

\subsection{Aufgabenstellung}
Machen Sie sich zunächst mit dem Befehl und den Parametern vertraut. (0,5 Punkte)

\subsection{Vorbereitung}
Für diese Aufgabe sollte man sich eine gute Wiki-Seite suchen, in unserem Fall benutzen wir die von ubuntuusers (\url{https://wiki.ubuntuusers.de/netstat/}).

\subsection{Durchführung}
netstat ist ein Diagnose-Werkzeug um Informationen über die Netzwerkschnittstelle zu bekommen.
\textbf{Syntax}: 
\begin{lstlisting}
$ netstat [OPTIONEN]
\end{lstlisting}
\textbf{Parameter}: 
\begin{table}[H]
	\tablestyle
	\begin{tabular}{lll}
		\toprule
			Option & Abkürzung & Beschreibung \tabularnewline
				
		\midrule
			- -tcp & -t & Zeigt nur TCP-Sockets an\tabularnewline
			- -udp & -t & Zeigt nur UDP-Sockets an\tabularnewline
			- -listening & -t & Zeigt nur offene Ports an\tabularnewline
			- -numeric & -n & Übersetzt keine Nummern in Namen\tabularnewline
			- -extend & -e & Zeigt erweiterte Informationen an\tabularnewline
			- -program & -p & Zeigt die Prozess-ID und den Programmnamen des Prozesses an\tabularnewline
			
	\end{tabular}
\end{table}
 
\subsection{Fazit}
Der \textit{netstat} ist ein simples Kommando und sollte niemanden schwerfallen.