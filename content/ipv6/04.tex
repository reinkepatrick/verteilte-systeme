\newpage

\section{Aufgabe 4}

\subsection{Aufgabenstellung}
Eine wichtige Neuerung von IPv6 ist die Fähigkeit der automatischen Selbstkonfiguration von Clients eines Netzsegmentes ohne einen konfigurierten DHCP-Server. Dieser Vorgang wird im \texttt{RFC 4862} als \texttt{Stateless Address Autoconfiguration (SLAAC)} beschrieben.

\begin{enumerate}[label=(\alph*)]
	\item Beschreiben Sie den Ablauf einer Autokonfiguration einer IPv6 Adresse für einen Client nach \texttt{SLAAC}.
	\item Welches Protokoll wird für \texttt{SLAAC} verwendet? Auf welchem Layer des OSI-Modells arbeitet es?
\end{enumerate}

\subsection{Vorbereitung}

\subsection{Durchführung}

\subsection{Fazit}