\newpage

\section{Aufgabe 4}

\subsection{Aufgabenstellung}
Eine wichtige Neuerung von IPv6 ist die Fähigkeit der automatischen Selbstkonfiguration von Clients eines Netzsegmentes ohne einen konfigurierten DHCP-Server. Dieser Vorgang wird im \texttt{RFC 4862} als \texttt{Stateless Address Autoconfiguration (SLAAC)} beschrieben.

\begin{enumerate}[label=(\alph*)]
	\item Beschreiben Sie den Ablauf einer Autokonfiguration einer IPv6 Adresse für einen Client nach \texttt{SLAAC}.
	\item Welches Protokoll wird für \texttt{SLAAC} verwendet? Auf welchem Layer des OSI-Modells arbeitet es?
\end{enumerate}

\subsection{Vorbereitung}
Einlesen in Stateless Address Autoconfiguration

\subsection{Durchführung}
a) Der Client bildet sich nach dem Start zunächst eine Link Local Adresse mit Interface Identifier aus der MAC-Adresse (Windows nutzt dafür nicht die MAC-Adresse, sondern eine zufällig generierte). Will dieser Client nun auf das Internet zugreifen, bekommt er per Router-Advertisement den globalen Präfix und bildet sich seine globale IPv6 Adresse selber.

b) Für diesen Vorgang wird das Neighbor Discovery Protocol (NDP) benutzt, mittels dem sich der Client auf die Suche nach einem Router macht. Es arbeitet auf Schicht 3 des OSI-Modells.
\subsection{Fazit}
Bei dieser Aufgabe gab es keine Probleme.
