\newpage

\section{Aufgabe 3}

\subsection{Aufgabenstellung}
Nachdem Sie sich mit dem Adressaufbau von IPv6 befasst haben, recherchieren Sie welche Adressbereiche von IPv6 für welche Aufgaben reserviert sind. Interessant ist hier die Frage woran man \texttt{Globale Unicast Adressen} erkennt, was eine \texttt{Link-Local Adresse} ist oder wie der \texttt{IPv6 Multicast} den IPv4 \texttt{Broadcast ersetzt}.

\begin{enumerate}[label=(\alph*)]
	\item Erläutern Sie kurz was \texttt{Globale Unicast Adressen} ausmacht, welche Adressbereiche dafür genutzt werden und von wem sie vergeben werden.
	\item Womit lässt sich eine \texttt{Unique Local Unicast} IPv6 Adresse aus der IPv4 Welt vergleichen?
	\item Geben Sie ein Beispiel für eine gültige  \texttt{Link Local Adresse}. Wofür wird die  \texttt{Link Local Adresse} genutzt und warum wird diese nicht für die reguläre Kommunikation zwischen z.B. Webserver und Webbrowser auf zwei Rechnern eines Subnetzes genutzt?
	\item Über welche \texttt{Multicast Adresse(n)} können alle Router eines Netzsegmentes erreicht werden?
	\item Welche Informationen (z.B. Subnetz, MAC-Adresse etc.) sind in der folgenden IPv6 Adresse enthalten?
\begin{lstlisting}
		2001:0db8:a513:8701:0a00:27ff:fe45:a0aa/64
\end{lstlisting}
\end{enumerate}

\subsection{Vorbereitung}
Für diese Aufgabe müssen wir uns mit \textbf{IPv6} auseinandersetzen.

\subsection{Durchführung}
\subsubsection{a)}
Adressbereiche:
\begin{itemize}
\item 0:0:0:0:0:ffff::/96 IPv4 mapped IPv6 Adressen. Verbindet "die alte Welt mit der neuen".
\item 2000::/3 sind von der IANA(Internet Assigned Numbers Authority) vergebene Adressen.
\item 2001-Adressen werden an Provider vergeben, die diese an ihre Kunden verteilen.
\item Mit 2003 beginende Adressen werden von der RIR(Regional Internet Registry) vergeben.
\end{itemize}

\subsubsection{b)}
Eine Unique Local Unicast Adresse ist vergleichbar mit den im eingenen Netzwerk (LAN) verteilten Adressen.

\subsubsection{c)}
Eine Link Local Adresse beginnt mit fe80:: oder febf::.

Es folgt ein Beispiel für eine Link Local Adresse:
\begin{lstlisting}
		fe80::468A:5Bff:fe9E:23F6
\end{lstlisting}

Eine Link Local Adresse ist nur im LAN erreichbar, wird also nicht vom Router weitergeleitet.

\subsubsection{d)}
Mit ff01::2,ff02::2 und ff05::2 können alle Router in einem Netzwerksegment angesprochen werden.

\subsubsection{e)}
MAC-Adresse: 08:00:27:45:a0:aa
Adressbereich (Subnetz): 2001:0db8:a513:8701:0:0:0:0 - 2001:0db8:a513:8701:ffff:ffff:ffff:ffff
Globale Multicast Adresse

\subsection{Fazit}
Bei dieser Aufgabe gab es keine Probleme.
