\newpage

\section{Aufgabe 2}

\subsection{Aufgabenstellung}
Machen Sie sich mit dem generellem Adressaufbau von IPv6 Adressen vertraut.
\begin{enumerate}[label=(\alph*)]
	\item Was versteht man unter Präfix und Interface-Identifier?
	\item  Stellen Sie die folgende IPv6 Adresse in maximal verkurzter Schreibweise dar:
\begin{lstlisting}
		2001:0db8:0000:090f:0000:0000:fe00:00aa/64
\end{lstlisting}
	\item Generieren Sie aus der MAC-Adresse einer ihrer Netzwerkkarten einen für IPv6 gültigen Interface-Identifier. Beschreiben Sie ihre Vorgehensweise.\\
	\textit{Hinweis: Modifiziertes EUI-64 Verfahren}
	\item  Informieren Sie sich über der CIDR Netznotation. Das Netz \texttt{2001:0db8:3210::/48} steht für welchen Adressbereich? Geben Sie die erste und die letzte Adresse an.
	\item Ihr privater Router bekommt von Ihrem Provider ein öffentliches IPv6-Netzsegment zugeteilt. Im Gegensatz zu IPv4 muss ihr Router keine\texttt{ Network Address Translation (NAT)} mehr nutzen um den Geräten Ihres internen LANs einen Zugriff auf das Internet zu ermöglichen. Warum?
	\item Nachdem Sie sich mit der Generierung von IPv6 Adressen befasst haben, treffen Sie eine Aussage über die Möglichkeit der Wiedererkennung eines Gerätes/Nutzers nach der Zuteilung eines neuen Netz-Präfixes (z.B. durch Standortwechsel des Endgeräts zwischen Arbeitsplatz und Privatanschluss). Wurde Sie die Situation als Problem betrachten und ggf. geeignete Gegenmaßnahmen ergreifen können?
\end{enumerate}

\subsection{Vorbereitung}

\subsection{Durchführung}

\subsection{Fazit}