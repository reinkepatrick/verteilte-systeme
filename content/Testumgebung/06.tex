\newpage

\section{Aufgabe 1: f}

\subsection{Aufgabenstellung}
Setzen Sie auf einer der virtuellen Maschinen die Software ownCloud auf (Ubuntu Installationspaket owncloud). Zeigen Sie, dass von der anderen virtuellen Maschine ein Zugriff auf ownCloud möglich ist. (2 Punkte)

\subsection{Vorbereitung}
Für die Installation von ownCloud wird ein Webserver mit Apache2, PHP und eine Datenbank benötigt. Wir führen diese Installation auf einem System mit \textit{Ubuntu 16.04} durch, welches wie wir oben eingerichtet haben.

\subsection{Durchführung}
Wir machen die komplette Installation zusammen, mit allen dazu gehörigen Paketen. 

\begin{enumerate}
	\item Als erstes aktualisieren Sie alle bereits vorhandenen Pakete
	\begin{lstlisting}
$ sudo apt-get update && sudo apt-get upgrade
	\end{lstlisting}
	\item Nun beginnen Sie mit der Installation von benötigten Paketen
	\begin{lstlisting}
$ sudo apt-get install -y apache2 mariadb-server 
libapache2-mod-php7.0 \
php7.0-gd php7.0-json php7.0-mysql 
php7.0-curl \ php7.0-intl php7.0-mcrypt 
php-imagick \ php7.0-zip php7.0-xml 
php7.0-mbstring
	\end{lstlisting}
	\item Jetzt wechseln Sie das Verzeichnis von ihrem frisch installiertem Webserver
	\begin{lstlisting}
$ cd /var/www
	\end{lstlisting}
	\item Nun laden Sie sich manuell die ownCloud runter, weil sie kann wegen Sicherheitsproblemen nicht mehr aus den offiziellen Paketquellen von Ubuntu installiert werden.
	\begin{lstlisting}
$ sudo wget https://download.owncloud.org/community/owncloud-10.0.3.tar.bz2
	\end{lstlisting}
	\item Jetzt entpacken Sie die gerade heruntergeladene Datei
	\begin{lstlisting}
$ sudo tar -xf owncloud-10.0.3.tar.bz2
	\end{lstlisting}
	\item Jetzt konfigurieren Sie noch den installierten Webserver
	\begin{lstlisting}
$ sudo nano /etc/apache2/sites-available/owncloud.conf
	\end{lstlisting}
	\begin{lstlisting}
Alias /owncloud "/var/www/owncloud/"
<Directory /var/www/owncloud/>
  	Options +FollowSymlinks
  	AllowOverride All

 	<IfModule mod_dav.c>
  		Dav off
 	</IfModule>

 	SetEnv HOME /var/www/owncloud
 	SetEnv HTTP_HOME /var/www/owncloud

</Directory>
	\end{lstlisting}
	\item Jetzt geben Sie dem www-data die notwendigen Rechte für die Installation
	\begin{lstlisting}
$ sudo chown -R www-data:www-data /var/www/owncloud/
	\end{lstlisting}
	\item Nun rufen Sie mit einem Browser ihrer Wahl die Webseite von ihrer persönlichen ownCloud auf (url{http://localhost/owncloud/}).
	\item Jetzt folgen Sie nur noch die Anweisungen die ihnen ownCloud gibt. \\\\
	\textbf{Hinweis}: Die Datenbankdaten lauten: \\ 
			\textit{Benutzername: root \\ 
			Passwort:}
\end{enumerate}

\subsection{Fazit}
Die Installation konnten wir nur theoretisch durchführen \cite{owncloud:install}, da wir mit diversen Ubuntu Distributionen Probleme hatten. \textit{Ubuntu 12.04} und \textit{Ubuntu 17.04} hatten Paketabhängigkeitsprobleme. Unter \textit{Ubuntu 16.04} hatten wir das Problem, dass er den Datenbank-Benutzer nicht angenommen hat, trotz mehrmaligen Passwortänderungen.