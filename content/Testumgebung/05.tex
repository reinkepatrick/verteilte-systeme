\newpage

\section{Aufgabe 1: e}

\subsection{Aufgabenstellung}
Informieren Sie sich über eine Software Ihrer Wahl, deren Funktionalität von Netzwerkunterstützung geprägt ist (z.B. Skype, BitTorrent, ownCloud). Beschreiben Sie die Software und ihre Funktionalität in einem kurzen, mit Ihren eigenen Worten formulierten Text von ca. 1500 Zeichen Länge. Tipp: vermeiden Sie bei dieser Aufgabe die Verwendung von Wikipedia. (2 Punkte)

\subsection{Vorbereitung}
Um diese Aufgabe bewältigen zu können benötigt man zuerst eine Software die man beschreiben kann, wenn man eine Software gefunden, auf die die Anforderungen zutreffen, muss deren Funktionalität beschrieben werden.

\subsection{Durchführung}
Unsere ausgewählte Software lautet ownCloud. \\\\ OwnCloud wurde im Januar 2010 gestartet und im KDE Camp von Frank Karlitschek vorgestellt \cite{owncloud:history}. Die angegebene Software ist ein Cloud-Speicher-Dienst, der ohne Drittanbieter funktioniert. Sie ist komplett OpenSource, sowohl Server als auch Client \cite{owncloud:github}. \\\\ Die Software läuft auf einen eigenen Server, wird von einem selbst installiert und bietet die Möglichkeit für jeden, dass an dem Projekt mitentwickeln kann. \\\\ Sie hat ein breites Funktionsspektrum, auf die wir alle einzeln eingehen werden. Die erste Funktion ist auch die wahrscheinlich wichtigste, das sichern und zugreifen auf Daten, von überall. Die hochgeladenen Dokumente kann man mit jedem teilen, wenn man diese teilen möchte. Das Teilen funktioniert mit einzelnen Personen und Gruppen, es gibt ein eigenes Rechtesystem um einer anderen Person oder Gruppe z.B das Recht zum teilen zu geben, damit er es ebenfalls mit anderen Personen teilen kann. Man hat die Möglichkeit hochgeladene Dateien zu kommentieren. Die ownCloud bietet eine eigene Desktop Applikation für Windows, Linux und MacOS. Es gibt auch eine iOS und Android Applikation. Eine weitere Funktion von ownCloud ist ein eingebauter Kalender und Kontakte, die wie die Daten von überall verfügbar sind. Es bietet dir die Option, deine Rich Text Dokumente mit anderen, in Echtzeit, zu bearbeiten. Du kannst dir PDFs, Bilder und Videos angucken. Es gibt dir auch für jede Änderung, die an deinen Dateien passiert, eine Benachrichtigung. OwnCloud hat ihre eigene Versionierung und du hast die Chance, auf die vorherige Version einer Datei zuzugreifen und Löschungen rückgängig zu machen. Zu guter Letzt hat man die Möglichkeit Passwörter zu speichern \cite{owncloud:features}.

\subsection{Fazit}
OwnCloud ist eine sehr umfangreiche Software. Sie bietet eine gute Funktionalität und ist eine Software für jeder Mann, weil es für jedem etwas bringt. 
