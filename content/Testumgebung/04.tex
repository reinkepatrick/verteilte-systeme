\newpage

\section{Aufgabe 1: d}

\subsection{Aufgabenstellung}
Prüfen Sie die Erreichbarkeit der Maschinen untereinander! Welches Programm verwenden Sie dazu? (1 Punkt)

\subsection{Vorbereitung}
Sie benötigen für die Aufgabe zwei virtuellen Maschinen mit installiertem \textit{Ubuntu 12.04} und konfiguriert in einem \textit{Host-only} Netzwerk, wie in den vorherigen Aufgaben beschrieben. Sie sollten diese gestartet haben.

\subsection{Durchführung}
Für diesen Test verwenden Sie \textit{ping}.

\begin{enumerate}
	\item Startet Sie mit \textit{studia} das Terminal mit der Tastenkombination Strg + Alt + T
	\item Geben Sie im Terminal \\
	\begin{lstlisting}
$ ping -c 3 192.168.56.102
	\end{lstlisting}
	\textbf{Hinweis}: Der Parameter \textit{-c} von \textit{Ping} steht für \textit{count}, er stoppt nachdem er die angegebene Anzahl an Requests gesendet hat. \\\\
	\textbf{Hinweis}: Die angegebene IP-Adresse gehört zu \textit{studib}
	\item Diese Versuch können Sie nun auch umgekehrt ausführen.
\end{enumerate}

\subsection{Fazit}
Diese Aufgabe ist leicht durchzuführen, wenn man die IP-Adressen selbst verteilt hat, sonst muss man diese vorher mit \textit{ifconfig} herausfinden. Diese Aufgabe hat keine Probleme bereitet.

